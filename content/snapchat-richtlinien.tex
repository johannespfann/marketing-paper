Snapchat besitzt Community-Richtlinien und Standards~\cite{sc_richtlinien} um
einen sicheren Umgang mit der Anwendung zu gew\"ahrleisten. Werden diese nicht
eingehalten, kann es zu Sperrung bzw. L\"oschung des Accounts f\"uhren. Die
wichtigsten Ans\"atze der Richtlinien werden im Folgenden kurz
zusammengetragen.

\paragraph{Allgemeines}
Es wird darauf hingewiesen, dass die Nutzer sich klar sein sollten, was sie
\"uber Snapchat teilen. Insbesondere wird nochmals explizit darauf hingewiesen,
dass der Gegen\"uber die M\"oglichkeit besitzt, gesendete Daten
\emph{abzufotographieren} oder andersweitig zu \emph{vervielf\"altigen}. Zudem
betont das Unternehmen, dass die Inhalte \emph{legal} sein sollten.

\paragraph{Verbotene Inhalte}
Snapchat weist explizit darauf hin, dass jede Art von \emph{Pornographie} in
ihrem Netzwerk verboten ist. Das beinhaltet ein Verbot zur Verbreitung von
pornographischen und nicht jugendfreien Inhalt, angedeutete sexuelle Handlung,
sowie Nacktheit in Verbindung mit sexuellen Handlungen.

Jugendliche sollen besonders gesch\"utzt werden. Deswegen untersagt Snapchat
das Versenden von \emph{Nacktdarstellungen oder sexuell aufreizende Inhalte von
Minderj\"ahrigen}.

Zudem weist das Unterehmen darauf hin, dass es verboten ist, Bilder oder Videos
von Dritten ohne deren Zustimmung zu versenden. Auch das Drohen bzw. Mobben
anderer Nutzer wird untersagt.
