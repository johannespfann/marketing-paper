Die größte Problematik besteht darin, das Nutzer nur wenig darauf achten, wie sie mit den neuen Medien umgehen. Speziell bei YouNow ergeben sich folgende Problemstellungen, welche auf der Internetseite \cite{KS15} vorgestellt werden.

\paragraph{Übertragung in Echtzeit}
Da die Übertragung in Echtzeit geschieht, folgt daraus die Problematik, das nichts zurückgenommen werden kann. Schnell kann passieren das persönliche Daten preisgegeben werden oder Meinungen, die unbedacht geäußert werden. Außerdem können viel Informationen durch emotionale Reaktionen preisgegeben werden, wenn beispielsweise auf bestimmte Fragen ungewöhnlich reagiert wird. Diese sind dann für alle sichtbar und können im Nachhinein nicht revidiert werden.

\paragraph{Persönlichkeitsrecht}
Es besteht die Möglichkeit, das im Hintergrund Personen sich aufhalten, denen nicht bewusst ist, das sich gerade in einem Live-Stream sind. Das ist jedoch ein Verstoß gegen das Recht am eigenen Bild. Ein weiteres Problem stellt eine Zeit-Autorin dar. Viele interviewen oder filmen ihre kleineren Geschwister, die sich nicht bewusst sein können, welche Konsequenzen YouNow für sie haben (siehe \cite{ZEIT15}).
Außerdem wird im Hintergrund oft Musik gespielt. Dies kann als Urheberrechtsverletzung gewertet werden. Eine Studie zeigt, dass 37 \% aller Broadcasts aus Deutschland gegen das Urheberrecht verstoßen (siehe \cite{HFMNF15}).

\paragraph{Jugendschutz}
Obwohl YouNow in seinen Regeln und bei der Anmeldung deutlich macht, dass sich keine Jugendlichen unter 13 Jahren anmelden dürfen, kann dies nicht kontrolliert werden. Es gibt Moderatoren, die bei Missachtung des Mindestalters diejenigen sperren, was aber nur schwer zu kontrollieren ist, wenn keine Angaben darüber gemacht werden. Außerdem kann aufgrund der Masse an Live-Streams nicht gewährleistet werden, dass alle anstößigen Inhalte auch zügig von den Moderatoren entfernt werden. Ein speziell für das Thema herausgebrachtes Schreiben (siehe \cite{YTD15}) von YouNow macht deutlich, das dies durchaus ein Problem für den Dienstanbieter darstellt. 

\paragraph{Ungewollten Kontaktaufnahme}
Da man als Moderator eines Live-Streams sein Publikum nicht sieht, kann man auch nicht wissen, wer unter einem Pseudonym chattet und welche Intention die jeweilige Person damit verfolgt. Außerdem kann man auch ohne Anmeldung einen Live-Streams beobachten.


