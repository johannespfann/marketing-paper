Nachfolgend werden einzelne Beiträge aus Pressemitteilungen aufgelistet die einen Einblick auf die Schattenseiten des YouNow-Protals zeigen sollen. Das dies durchaus ein Problem für den Betreiber YouNow darstellt, zeigt der Artikel von YouNow, in dem dieser verstärkt Moderatoren für die Einhaltung der Regeln einstellen wolle und er sich verstärkt um Minderjährige Nutzer kümmern möchte (siehe \cite{YTD15}).

\paragraph{Weiblicher Fan wird vor der Kamera zum Ausziehen überredet}
Ein sehr harter Vorfall zeigt, wie ein Nutzer einen anderen weiblichen Fan befragt, ob sie mit ihm Geschlechtsverkehr haben möchte. Als sie dieses bejaht, geht er einen Schritt weiter und fragt diese, ob sie auch Oralverkehr mache. Nach einer weiteren Bejahung fordert er sie dazu auf sich vor der Kamera auszuziehen. Dabei gibt er vor, alle anderen aus dem Chat geblockt zu haben, damit keiner etwas sieht. Nachdem der weibliche Fan auch noch den BH auszieht, gibt er zu, das jeder in seinem Live-Stream zugesehen habe (siehe \cite{MD10}).

\paragraph{Stern-Online warnt vor YouNow}
In einem Artikel im Stern (siehe \cite{STERN15}) wird an die Eltern appelliert mehr in Erfahrung zu bringen, welche Medien ihre Kinder nutzen. Diese sollten das Gespräch suchen und vor den Gefahren aufklären. Zu sehen ist in dem Artikel ein weiblicher Nutzer mit einem Zettel auf dem steht: \glqq Bei 200 Likes könnt ihr mich im BH sehen\grqq .

\paragraph{Rechtsanwälte klären auf}
Die Rechtsanwaltskanzlei WILDE BEUGER SOMECKE warnt vor den Rechtlichen Fallstricken des YouNow-Portals (siehe \cite{WBS15}). Hier wird darauf aufmerksam gemacht, dass das Filmen von Unterricht nicht legal sei. Außerdem entstehen Urheberrechtsverletzungen, weil Nutzer im Hintergrund Musik laufen ließen. Ein Studie zeigte auf, das 37 \% aller Broadcasts aus Deutschland gegen das Urheberrecht. Weitere 12 \% gegen Persönlichkeitsrecht, 8 \% enthalten Beleidigungen oder zeigen Drogenkonsum von Minderjährigen (siehe \cite{HFMNF15}).

