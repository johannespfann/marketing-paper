YouNow ist eine kostenlose Live-Videostreaming-Plattform, die als App auf dem Smartphone oder als Desktop-Anwendung verfügbar ist. Nutzer können nach einer Anmeldung mit der Kamera ihres Smartphones, Laptop oder Tablets Live-Streams von sich aufnehmen und für andere Nutzer bereitstellen oder bei dessen Live-Streams zusehen. Neben der Möglichkeit des Live-Streams bietet die Plattform einen Chat an, sodass in einem Live-Stream alle Beteiligten miteinander kommunizieren können. Außerdem kann ein weiterer Nutzer sich an einem Video beteiligen. 

\paragraph{Anmeldung}
Nach Installation der App auf dem Smartphone oder dem Öffnen der Website mit dem Browser kann man sich mit Facebook, Gmail- oder Twitter-Account anmelden. Diese Anmeldung erfordert allerdings sämtliche Kontaktdaten und Email-Adressen des Kontoinhabers. Außerdem muss man versichern, das man nicht jünger als 13 Jahre alt ist.
Im nächsten Schritt wird man aufgefordert einen Namen anzugeben, mit dem man in Videos und Chats erkannt wird.

\paragraph{Kommunikation}
Während des Live-Streams ist gleichzeitig ein Chat aktiv, mit dem sich der Protagonist mit seinen Zuschauern unterhalten kann. Chatteilnehmer können sich Nachrichten, Sticker oder Icons zusenden. Neben den Chats, ermöglicht die App das Versenden von Posts zu einzelnen Personen. Diese erscheinen, ähnlich wie bei der Facebook-Chronik, auf einer Pinnwand der jeweiligen Person. Außerdem können sich YouNow-Nutzer private Nachrichten zusenden.


\paragraph{Geschenke und Währung}
Bei YouNow gibt es zwei verschiedene Währungen die jeweils für verschiedene Aktionen gedacht sind. Zum einen gibt es Münzen (Coints), die repräsentieren einen Wert eines Geschenkes. Mit einem Geschenk, kann ein Zuschauer auf sich aufmerksam machen bzw. den Stream eines anderen wertschätzen, indem er ein Geschenk für diesen versendet. Geschenke sind in diesem Fall kleine Bilder (Icons).
Coints haben keinen realen Gegenwert und müssen in YouNow verdient werden. Mit folgenden Aktionen können Coints verdient werden:

\begin{itemize}
	\item Veröffentlichen von Livestreams
	\item Aktive Teilnahme. Durch das Anschauen von Livestreams, Liken oder chatten kann man Münzen verdienen.
	\item Deine Freunde für eine Anmeldung bei YouNow werben. Für jede Anmeldung deiner Freunde über einen Facebook-Post, Tweet, E-Mail, Tumblr-Verlinkung oder Facebook-Einladung von dir erhältst Du Münzen. 
	\item Dich jeden Tag anmelden.
\end{itemize} 

Eine weitere Währung sind die sog. \textbf{bars}. Diese können nicht durch Aktionen erworben werden, sondern müssen gekauft werden. Die haben also einen realen Gegenwert. Mit diesen können dann Premiumgeschenke gekauft werden, mit denen man sich in einem Chat besonders auf sich aufmerksam machen kann. Premiumgeschenke könnten sein:

\begin{itemize}
	\item 50 Likes: Ein Geschenk, dass den Broadcaster hilft zu trenden
	\item Fanpost: Eine personalisierte Chat-Nachricht, die heraussticht 
	\item 50 Likes für den eigenen Stream: Hilft dem Broadcaster, mehr Aufmerksamkeit zu erlangen
	\item Chat-cool-down-bypass: Aufhebung der Begrenzung der eigenen Chatnachrichten in vollen, überlasteten Chats
	\item Bars: Um Partner unmittelbar zu unterstützen
\end{itemize}

\paragraph{Levelsystem}
YouNow führt außerdem ein Levelsystem ein. Dieses hat die Aufgabe den Nutzer dazu anzuregen mehr Aktionen auf YouNow zu tätigen. Der Anreiz um weitere Levels zu bekommen ist, dass der Nutzer dadurch Zugriff auf bessere und speziellere Geschenke erlangt um den Streamenden zu unterstützen. Außerdem steigt pro Level auch der Wert an Münzen, die dann auf die normalen Geschenke berechnet werden. 
Um ein höheres Level zu erreichen hat der Nutzer folgende Möglichkeiten.

\begin{itemize}
	\item je mehr Geschenke sie selbst erhalten
	\item je mehr Likes sie erhalten
	\item je größer ihr Publikum ist
	\item je mehr Live-Streams sie tätigen
	\item je mehr sie mit anderen Nutzern chatten
	\item je mehr Geschenke sie selbst vergeben
	\item je mehr Freunde sie zu YouNow einladen.
	\item und, wenn sie ihre Konten aus anderen Sozialen Netzwerken, wie z. B. YouTube, Facebook, Twitter mit ihrem YouNow-Konto verknüpfen.
\end{itemize}
