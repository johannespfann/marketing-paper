Auf der offiziellen Internetseite von
Snapchat\footnote{\url{https://www.snapchat.com}} propagiert das Unternehmen
den hohen Stellenwert von Datenschutz. Jedoch werden bei der Nutzung von
Snapchat und seinen Services Daten von Nutzern gesammelt. Um eine hohe
Transparenz zu gew\"ahrleisten, beschreibt das Unternehmen, welche Daten zu
welchen Zweck erhoben werden. Das Unternehmen unterscheidet in folgende
Gruppen:
\begin{enumerate}
	\item Vom Nutzer freiwillig zur Verf\"ugung gestellte Daten
	\item Daten, die durch die Nutzung der Services anfallen
	\item Daten durch Dritte
\end{enumerate}
Im folgenden gehen wir auf die \emph{Datenschutzbestimmungen} etwas
n\"aher ein.

\paragraph{Vom Nutzer freiwillig gestellte Daten}
Um Snapchat nutzen zu k\"onnen braucht man einen Nutzeraccount f\"ur dessen
Erstellung man Informationen wie Name, Email-Adresse, Passwort, Alter und ggf.
Rufnummer angeben muss. Sein Profil kann man zudem noch weiter bearbeiten,
indem man ein Profilbild oder andere hilfreiche Informationen von sich
preisgibt. Diese Daten werden vom Unternehmen gespeichert. Kommerzielle Dienste
innerhalb der App ben\"otigen ggf.  Kreditkartennummern oder andere
Zahlungsinformationen. Deine gesendeten Bilder und Videos k\"onnen zum einen
vom Empf\"anger gespeichert werden (z.B. durch Screenshot) und werden ebenfalls
auch von Snapchat gespeichert.

\paragraph{Daten, durch das Benutzen von Services}
Die Anwendung sammelt eine Vielzahl von Daten, die durch das Benutzen der
Services anfallen. Das beinhaltet \emph{Nutzungsdaten} wie z.B. wie man mit
anderen Personen kommuniziert oder wie der User bestimmte Services nutzt.
Au{\ss}erdem werden die Inhalte jeder Kommunikation sowie ger\"atespezifische
Daten wie Telefonbuch, Kamera, Gallery und Standortdaten gespeichert.

\paragraph{Daten von Dritten}
Snapchat kann Daten erfassen, die von anderen Nutzern \"uber einen bestimmten
preisgegeben werden. Zum Beispiel, wenn ein bestimmter Nutzer A im Telefonbuch
eines Nutzers B aufgef\"uhrt wird, kann Snapchat diese Kontaktdaten mit den
Daten von Nutzer A zusammenf\"uhren.

\paragraph{Nutzung von Daten}
Das Unternehmen verabeitet die Daten demhingegehend, dass sie zum einen mit
ihren Nutzern kommunizieren und die eigentliche Anwendung weiterentwickeln
k\"onnen. Zum anderen wird aus den Daten personalisierte Werbung verbessert,
angepasst und ausgeliefert.

\paragraph{Weitergabe von Daten}
Die Daten von Nutzern kann von Snapchat auf folgende Weise weitergegeben werden:
\begin{itemize}
	\item \emph{An andere Snapchat-User}: Daten zu einer Person wie Name und
		Alter als auch weitere Daten, die der Nutzer freiwillig abgibt,
		k\"onnen an andere Nutzer weitergegeben werden, um bestimmte Personen
		schneller zu finden.
	\item \emph{An die \"Offentlichkeit}: Profilbilder, Snapcodes, und Medien,
		die ein Nutzer an \"offentliche Channels sendet.
	\item \emph{An Dritte}: Snapchat finanziert sich durch Werbung.
		Dementsprechend werden die personenbezogenen Daten an Drittanbieter aus
		der Werbebranche weitergegeben. Auch beh\"alt sich Snapchat das Recht,
		Daten aus rechtlichen Gr\"unden bei Missbrauch oder anderen
		Verst\"o{\ss}en offenzulegen.
	\item \emph{Aggregierte und anonymisierte Date}: Diese Daten werden
		ebenfalls an Werbeunternehmen weitergegeben.
\end{itemize}
