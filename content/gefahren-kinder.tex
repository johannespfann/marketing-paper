In Kapitel~\ref{einleitung} haben wir bereits erw\"ahnt, dass soziale Netzwerke
bei Kindern und Jugendlichen sehr beliebt sind. In diesem Abschnitt wollen wir
deren Gefahren, insbesondere die der P\"adophilie genauer betrachten. Hierf\"ur
konzentrieren wir uns in dieser Arbeit auf zwei m\"ogliche Szenarien (siehe
Abschnitt~\ref{gefahren:kinder:informationen}
und~\ref{gefahren:kinder:treffen}) und unterlegen diese mit realen
Fallbeispielen.

\subsubsection{Informationen beschaffen/sammeln}
\label{gefahren:kinder:informationen}
Die erste Problematik, die wir hier betrachten m\"ochten, ist das Beschaffen von
Informationen, die der Nutzer freiwillig bzw. unbewusst zur Verf\"ugung stellt.

Inbesondere bei YouNow, aber auch bei Snapchat, geschieht die \"Ubertragung
teilweise in Echtzeit, was zur Folge hat, dass die gegebenen Informationen und
Inhalte nicht zur\"uckgenommen werden k\"onnen. Somit kann es durchaus
vorkommen, dass pers\"onliche Daten preisgegeben oder Meinungen unbedacht
ge\"au{\ss}ert werden. Au{\ss}erdem k\"onnen viele Informationen durch
emotionale Reaktionen preisgegeben werden, wenn beispielsweise auf bestimmte
Fragen ungew\"ohnlich reagiert wird. Diese sind dann f\"ur alle sichtbar und
k\"onnen im Nachhinein nicht revidiert werden~\cite{KS15}.

Auch werden Bilder und Snaps ohne Aufforderung ver\"offentlicht, ohne darauf zu
achten, welche ungewollten Informationen damit \"ubertragen werden.
Beispielsweise k\"onnen hierbei Angreifer R\"uckschl\"usse daraus ziehen, wo
der Nutzer wohnt, oder in welchem (famili\"aren) Umfeld sich dieser befindet.
Wenn im Hintergrund Musik l\"auft, kann ein Angreifer ebenso den Musikgeschmack
seines Opfers kennenlernen und dies ggf. f\"ur weitere Aktionen benutzen
(siehe~\ref{gefahren:kinder:treffen}).

Abschlie{\ss}end kann man festhalten, dass sich daraus zwei Probleme ergeben.
Zum einen reicht es oftmals P\"adophilien Bild- und Videomaterial von
Minderj\"ahrigen zu besitzen und diese f\"ur ihre Fantasien zu benutzen. Zum
anderen werden oftmals zus\"atzliche Informationen weitergegeben, mit denen dem
Angreifer weitere Schritte wie der Kontaktaufnahme oder Stalking erm\"oglicht
werden.

\subsubsection{Kontakt aufnehmen}
\label{gefahren:kinder:treffen}
