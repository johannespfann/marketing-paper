In Kapitel~\ref{einleitung} haben wir bereits erw\"ahnt, dass soziale Netzwerke
bei Kindern und Jugendlichen sehr beliebt sind. In diesem Abschnitt wollen wir
deren Gefahren, insbesondere die der P\"adophilie genauer betrachten. Hierf\"ur
konzentrieren wir uns in dieser Arbeit auf zwei m\"ogliche Szenarien (siehe
Abschnitt~\ref{gefahren:kinder:informationen}
und~\ref{gefahren:kinder:treffen}) und unterlegen diese mit realen
Fallbeispielen.

\subsubsection{Informationen beschaffen/sammeln}
\label{gefahren:kinder:informationen}
Die erste Gefahr, die wir hier betrachten m\"ochten, ist das Beschaffen von
Informationen, die der Nutzer freiwillig bzw. unbewusst zur Verf\"ugung stellt.


\subsubsection{Kontakt aufnehmen}
\label{gefahren:kinder:treffen}
