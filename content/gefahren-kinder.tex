In Kapitel~\ref{einleitung} haben wir bereits erw\"ahnt, dass soziale Netzwerke
bei Kindern und Jugendlichen sehr beliebt sind. In diesem Abschnitt wollen wir
deren Gefahren, insbesondere die der P\"adophilie genauer betrachten. Hierf\"ur
konzentrieren wir uns in dieser Arbeit auf zwei m\"ogliche Szenarien (siehe
Abschnitt~\ref{gefahren:kinder:informationen}
und~\ref{gefahren:kinder:treffen}) und unterlegen diese mit realen
Fallbeispielen.


\subsubsection{Allgemein}
\label{gefahren:kinder:allgemein}
Zun\"achst wollen wir in diesem Abschnitt einige allgemeine Problematiken von
Snapchat und YouNow, die die Verwundbarkeit von Nutzern erh\"ohen, erl\"autern.

Ein gro{\ss}er Kritipunkt bei beiden Anwendungen ist das Thema
\emph{Jugendschutz}. Obwohl YouNow und Snapchat in ihren Gesch\"aftsbedinungen
angeben, dass Jugendlichen unter 13 Jahren die Anmeldung untersagt ist, wird
dies nicht kontrolliert. Beispielsweise wird bei der Anmeldung ein
Geburstsdatum verlangt, was aber nicht verifiziert wird, und durch ein
beliebiges Datum ersetz werden kann. YouNow versucht beispielsweise dagegen
vorzugehen, indem es Moderatoren einstellt. Diese agieren zum Schutz der Nutzer
und sperren Accounts, die das Mindestalter nicht einhalten, oder entfernen
anst\"o{\ss}ige Inhalte. Diese Verfahren ist aber aufgrund der schweren
Verifizierung und der Masse an Streams schwer durchf\"uhrbar.

Besonders bei YouNow f\"allt auf, dass \emph{jeder}, auch wenn man nicht
registriert ist, auf Streams zugreifen kann, d.h. der Nutzer, der das Video
einstellt, hat keine Kontrolle dar\"uber, wer das Video ansieht. Ein weiteres
Problem ist die Anonymit\"at, die ein solcher Dienst seinen Nutzern zur
Verf\"ugung stellt. Beispielsweise kann man bei YouNow nicht wissen, welche
Person sich hinter einem bestimmten Pseudonym verbirgt.

Diese erw\"ahnten Schwachstellen k\"onnen und werden von Angreifern bzw.
P\"adophilen ausgenutzt (siehe Abschnitt~\ref{gefahren:kinder:informationen}
und~\ref{gefahren:kinder:treffen}).

\subsubsection{Informationen beschaffen/sammeln}
\label{gefahren:kinder:informationen}
Die erste Problematik, die wir hier betrachten m\"ochten, ist das Beschaffen von
Informationen, die der Nutzer freiwillig bzw. unbewusst zur Verf\"ugung stellt.

Inbesondere bei YouNow, aber auch bei Snapchat, geschieht die \"Ubertragung
teilweise in Echtzeit, was zur Folge hat, dass die gegebenen Informationen und
Inhalte nicht zur\"uckgenommen werden k\"onnen. Somit kann es durchaus
vorkommen, dass pers\"onliche Daten preisgegeben oder Meinungen unbedacht
ge\"au{\ss}ert werden. Au{\ss}erdem k\"onnen viele Informationen durch
emotionale Reaktionen preisgegeben werden, wenn beispielsweise auf bestimmte
Fragen ungew\"ohnlich reagiert wird. Diese sind dann f\"ur alle sichtbar und
k\"onnen im Nachhinein nicht revidiert werden~\cite{KS15}.

Auch werden Bilder und Snaps ohne Aufforderung ver\"offentlicht, ohne darauf zu
achten, welche ungewollten Informationen damit \"ubertragen werden.
Beispielsweise k\"onnen hierbei Angreifer R\"uckschl\"usse daraus ziehen, wo
der Nutzer wohnt, oder in welchem (famili\"aren) Umfeld sich dieser befindet.
Wenn im Hintergrund Musik l\"auft, kann ein Angreifer ebenso den Musikgeschmack
seines Opfers kennenlernen und dies ggf. f\"ur weitere Aktionen benutzen
(siehe~\ref{gefahren:kinder:treffen}).

Abschlie{\ss}end kann man festhalten, dass sich daraus zwei Probleme ergeben.
Zum einen reicht es oftmals P\"adophilien Bild- und Videomaterial von
Minderj\"ahrigen zu besitzen und diese f\"ur ihre Fantasien zu benutzen. Zum
anderen werden oftmals zus\"atzliche Informationen weitergegeben, mit denen dem
Angreifer weitere Schritte wie der Kontaktaufnahme und/oder dem Aufbauen einer
Beziehung erm\"oglicht werden (siehe~\ref{gefahren:kinder:treffen}).

\subsubsection{Kontakt aufnehmen}
\label{gefahren:kinder:treffen}
Im vorherigen Abschnitt haben wir gesehen, dass es Angreifern sehr einfach
gemacht wird, Informationen \"uber ihre Opfer zu sammeln. In diesem Abschnitt
wollen wir uns n\"aher mit den Aktionen befassen, die zum Teil diese
gesammelten Informationen nutzen, um Kontakt mit den Nutzern herzustellen.
