Wir haben in den Abschnitten~\ref{yn:datenschutz} und~\ref{sc:datenschutz}
aufgezeigt, welche Daten von YouNow bzw. Snapchat gesammelt und ggf.
weitergegeben werden. In diesem Abschnitt wollen wir deren Umgang mit Daten
kritisch hinterfragen.

Snapchat gibt in seinen ABGs an, dass die Daten, die ein User verschickt, nur
f\"ur einen bestimmten Zeitraum auf dem Endger\"at des Empf\"angers vorhanden
und anschlie{\ss}end gel\"oscht werden. In der Praxis sieht das leider etwas
anders aus: Die Dateien werden lediglich mit einer anderen Endung versehen,
damit Gallerie-Apps diese nicht mehr finden und anzeigen k\"onnen. Somit
bleiben die Inhalte auf dem Ger\"at vorhanden und k\"onnen mit etwas Aufwand
wiederhergestellt werden~\cite{sc_risiken}.

Was man auch noch bedenken muss, wenn man private Bilder via Snapchat
versendet, mit dem Hintergedanken, dass sie nur kurz f\"ur den Empf\"anger
sichtbar sind, ist, dass der Empf\"anger immer noch die M\"oglichkeit besitzt,
die Inhalte per Screenshot oder mit einem zus\"atzlichen Aufnahmeger\"at
sichern kann~\cite{sc_risiken}.

Ein weiterer Kritikpunkt ist Snapchats Umgang mit den Daten seiner User. Wie
bereits erw\"ahnt sammelt das Unternehmen eine Reihe personenbezogener Daten
und der Nutzer willigt automatisch den Datennutzungsbestimmungen (siehe oben)
und damit verbundene Weiterverarbeitung und -gabe zu. Medieninhalte wie Bilder
und Videos werden bis zum Abruf vom Empf\"anger auf den Servern, welche in den
Vereinigten Staaten stehen, gespeichert. Innerhalb dieses Zeitraums hat
Snapchat vollen und freien Zugriff auf die Bilddateien seiner
User~\cite{sc_risiken}.
