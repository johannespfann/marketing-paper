In den letzten Kapiteln haben wir die beiden sozialen Medien Snapchat und
YouNow etwas genauer betrachtet. Zun\"achst wurden die beiden Dienste
ausf\"uhrlich im Bezug auf Funktionsweise und Richtlinien beschrieben.
Anschliessend hat sich diese Arbeit kritisch mit den Gefahren f\"ur Kinder und
Jugendliche, die in solchen Netzwerken agieren, befasst. Insbesondere wurden
hier neben datenschutzrechtlichen Bedenken vor allem deren Anf\"alligkeit
gegen\"uber P\"adophilen herausgearbeitet. Es wurden Beispiele aus der Presse
vorgestellt, die diese Problematik unterstreichen und die Gefahren Realit\"at
werden lassen. Daraufhin wurden noch Hinweise und Tipps f\"ur den
aufmerksamen Umgang mit diesen Medien vorgestellt und zusammengefasst.


Abschlie{\ss}end l\"asst sich festhalten, dass die angesprochenen Gefahren real
und allgegenw\"artig sind. Wie in Kapitel~\ref{einleitung} erw\"ahnt, sind
soziale Medien inbesondere bei Kindern und Jugendlichen stark vertreten und
werden dies auch die kommenden Jahren sein. Das hat zur Folge, dass besonders
diese Nutzergruppe den Gefahren ausgesetzt sein wird. Es gibt bereits mehrere
Internetseiten, die versuchen darauf zu reagieren, indem sie dem Nutzer
verschiedene Tipps und Hinweise f\"ur den korrekten und sicheren Umgang mit
solchen Medien aufzeigen. Auch Eltern sind dahingehend gefordert, dass sie
ihren Kindern einen sensiblen Umgang mit neuen Medien vorleben und zeigen. Ein
anderer Ansatz wird seit ein paar Jahren in der Politik und den
Erziehungswissenschaften diskutiert. N\"amlich die Einf\"uhrung eines
Schulfachs \emph{Medienkompetenz}, das Sch\"uler einen bewussten Umgang im
Internet beibringen soll.
